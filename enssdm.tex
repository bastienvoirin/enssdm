\documentclass{article}
\usepackage[margin=2cm]{geometry}
\usepackage[fr,code,showframes,longto,longmapsto,noindent,widehat]{enssdm}

\title{Package \raw{enssdm}}
\author{Bastien Voirin}

\begin{document}
\maketitle

\begin{abstract}
    Le package \raw{enssdm} rassemble des packages et définit des commandes utiles en Sciences de la Matière et plus généralement en Sciences Exactes et Expérimentales : mise en page, notations scientifiques, symboles, environnements, écriture de codes sources et d'algorithmes...
\end{abstract}

\tableofcontents

\clearpage
\section{Options disponibles}

\begin{xltabular}[l]{0.825\textwidth}{@{} lX @{}}
\raw{fr} & Définit le français comme langue principale du document. Affecte aussi certaines notations mathématiques : vecteurs $\vec{v}$, produit vectoriel $\vec{A}\wedge\vec{B}$...\\\\
\raw{en} & Définit l'anglais comme langue principale du document. Affecte aussi certaines notations mathématiques : vecteurs $\mathbf{v}$, produit vectoriel $\mathbf{A}\cross\mathbf{B}$...\\\\
\raw{code} & Importe des packages et définit des commandes utiles pour afficher du code source et écrire des algorithmes dans le document.\\\\
\raw{showframes} & Options de débogage qui dessine des cadres autour de certaines boîtes pour rendre visible l'espacement, la disposition et l'alignement de ces boîtes.\\\\
\raw{widehat} & Change la commande \raw{\hat} ($\,\oldhat{\:}\,$ par défaut) en $\,\widehat{\:}\,$. La première variante est toujours accessible grâce à la commande \raw{\oldhat}.\\\\
\raw{longto} & Change la commande \raw{\to} ($\oldto$ par défaut) en $\longrightarrow$. La première variante est toujours accessible grâce à la commande \raw{\oldto}.\\\\
\raw{longmapsto} & Change la commande \raw{\mapsto} ($\oldmapsto$ par défaut) en $\longmapsto$.  La première variante est toujours accessible grâce à la commande \raw{\oldmapsto}.
\end{xltabular}

\section{Packages disponibles}

Liste non exhaustive des packages importés par \raw{enssdm} :

\begin{table}[H]
    \centering
    \begin{tabular}{ll}
        \href{https://www.ctan.org/pkg/adjustbox}{\raw{adjustbox}} & <<~\emph{Graphics package-alike macros for ``general'' boxes}~>>\\
        \href{https://www.ctan.org/pkg/amsfonts}{\raw{amsfonts}} & <<~\emph{{\normalfont\TeX} fonts from the American Mathematical Society}~>>\\
        \href{https://www.ctan.org/pkg/amsmath}{\raw{amsmath}} & <<~\emph{AMS Mathematical facilities for {\normalfont\LaTeX}}~>>\\
        \raw{amssymb} & \\
        \href{https://www.ctan.org/pkg/booktabs}{\raw{booktabs}} & <<~\emph{Publication quality tables in {\normalfont\LaTeX}}~>>\\
        \href{https://www.ctan.org/pkg/enumitem}{\raw{enumitem}} & <<~\emph{Control layout of itemize, enumerate, description}~>>\\
        \href{https://www.ctan.org/pkg/esint}{\raw{esint}} & <<~\emph{Extented set of integrals for Computer Modern}~>>\\
        \href{https://www.ctan.org/pkg/esvect}{\raw{esvect}} & <<~\emph{Vector arrows}~>>\\
        \href{https://www.ctan.org/pkg/float}{\raw{float}} & <<~\emph{Improved interface for floating objects}~>>\\
        \href{https://www.ctan.org/pkg/mathtools}{\raw{mathtools}} & <<~\emph{Mathematical tools to use width {\normalfont\raw{amsmath}}}~>>\\
        \href{https://www.ctan.org/pkg/mhchem}{\raw{mhchem}} & <<~\emph{Typeset chemical formulae/equations and Risk and Safety phrases}~>>\\
        \href{https://www.ctan.org/pkg/nicematrix}{\raw{nicematrix}} & <<~\emph{Improve the typesetting of mathematical matrices with \textsc{pgf}}~>>\\
        \href{https://www.ctan.org/pkg/pgf}{\raw{pgf} (et TikZ)} & <<~\emph{Create PostScript and PDF graphics in {\normalfont\TeX}}~>>\\
        \href{https://www.ctan.org/pkg/physics}{\raw{physics}} & <<~\emph{Macros supporting the Mathematics of Physics}~>>\\
        \href{https://www.ctan.org/pkg/siunitx}{\raw{siunitx}} & <<~\emph{A comprehensive (SI) units package}~>>
    \end{tabular}
\end{table}

\clearpage
\section{Mise en page}

\subsection{Listes, tableaux, environnements}

\subsubsection{Listes non numérotées}

Le package \raw{enssdm} contient le code suivant, qui peut être écrasé (copié-collé puis modifié) dans le document \LaTeX :

\begin{minted}{latex}
\setlist[itemize,1]{label={\textbullet}}
\setlist[itemize,2]{label={\normalfont\bfseries\textendash}}
\setlist[itemize,3]{label={\textasteriskcentered}}
\setlist[itemize,4]{label={\textperiodcentered}}
\end{minted}

Ce code produit le résultat suivant :

\begin{itemize}
    \item Liste de niveau 1
    \begin{itemize}
        \item Liste de niveau 2
        \begin{itemize}
            \item Liste de niveau 3
            \begin{itemize}
                \item Liste de niveau 4
            \end{itemize}
        \end{itemize}
    \end{itemize}
\end{itemize}

En écrivant

\begin{minted}{latex}
\begin{itemize}
    \item Liste de niveau 1
    \begin{itemize}
        \item Liste de niveau 2
        \begin{itemize}
            \item Liste de niveau 3
            \begin{itemize}
                \item Liste de niveau 4
            \end{itemize}
        \end{itemize}
    \end{itemize}
\end{itemize}
\end{minted}

\subsubsection{Listes numérotées}

\begin{enumerate}
    \item Liste de niveau 1
    \begin{enumerate}
        \item Liste de niveau 2
        \begin{enumerate}
            \item Liste de niveau 3
            \begin{enumerate}
                \item Liste de niveau 4
            \end{enumerate}
        \end{enumerate}
    \end{enumerate}
\end{enumerate}

\begin{minted}{latex}
\begin{enumerate}
    \item Liste de niveau 1
    \begin{enumerate}
        \item Liste de niveau 2
        \begin{enumerate}
            \item Liste de niveau 3
            \begin{enumerate}
                \item Liste de niveau 4
            \end{enumerate}
        \end{enumerate}
    \end{enumerate}
\end{enumerate}
\end{minted}

\clearpage
\subsubsection{Double colonne}

Colonnes alignées en haut :

\twocol{
    \begin{align}
        \SwapAboveDisplaySkip
        E &= mc^2
    \end{align}
}{
    \begin{align}
        \SwapAboveDisplaySkip
        i^2 &= -1\\
        j^2 &= -1\\
        k^2 &= -1
    \end{align}
}

\begin{minted}{latex}
\twocol{
    % colonne de gauche
}{
    % colonne de droite
}
\end{minted}

\begin{minted}{latex}
\twocol[t]{
    % colonne de gauche
}{
    % colonne de droite
}
\end{minted}

Colonnes alignées au centre :

\twocol[c]{
    \begin{align}
        \SwapAboveDisplaySkip
        E &= mc^2
    \end{align}
}{
    \begin{align}
        \SwapAboveDisplaySkip
        i^2 &= -1\\
        j^2 &= -1\\
        k^2 &= -1
    \end{align}
}

\begin{minted}{latex}
\twocol[c]{
    % colonne de gauche
}{
    % colonne de droite
}
\end{minted}

Colonnes alignées en bas :

\twocol[b]{
    \begin{align}
        \SwapAboveDisplaySkip
        E &= mc^2
    \end{align}
}{
    \begin{align}
        \SwapAboveDisplaySkip
        i^2 &= -1\\
        j^2 &= -1\\
        k^2 &= -1
    \end{align}
}

\begin{minted}{latex}
\twocol[b]{
    % colonne de gauche
}{
    % colonne de droite
}
\end{minted}

\clearpage
\section{Sciences}

\subsection{Notations, symboles et constantes}

\begin{table}[H]
    \centering
    \begin{tabular}{ll ll ll}
    $\argmax$ & \raw{\argmax} &&& $\Argmax$ & \raw{\Argmax}\\
    $\argmin$ & \raw{\argmin} &&& $\Argmin$ & \raw{\Argmin}\\
    $\img$ & \raw{\img} &&& $\Img$ & \raw{\Img}\\
    $\kr$ & \raw{\kr} &&& $\Kr$ & \raw{\Kr}\\
    $\mat$ & \raw{\mat} &&& $\Mat$ & \raw{\Mat}\\
    $\per$ & \raw{\per} &&& $\Per$ & \raw{\Per}\\
    $\perm$ & \raw{\perm} &&& $\Perm$ & \raw{\Perm}\\
    $\rg$ & \raw{\rg} &&& $\Rg$ & \raw{\Rg}\\
    $\rk$ & \raw{\rk} &&& $\Rk$ & \raw{\Rk}\\
    $\spc$ & \raw{\spc} &&& $\Spc$ & \raw{\Spc}\\
    $\spn$ & \raw{\spn} &&& $\Spn$ & \raw{\Spn}\\
    $\vect$ & \raw{\vect} &&& $\Vect$ & \raw{\Vect}\\
    \end{tabular}
\end{table}

\begin{table}[H]
    \centering
    \begin{tabular}{llll}
    Nom(s) & Résultat & \multicolumn{2}{c}{Commande}\\
    \toprule
    Permittivité du vide & $\epz$ & \raw{\epz}\\
    Perméabilité du vide & $\muz$ & \raw{\muz}\\
    Constante d'Avogadro & $\na$ & \raw{\na}\\
    Magnéton de Bohr & $\mub$ & \raw{\mub}\\
    Constante de Boltzmann & $\kb$ & \raw{\kb}\\
    & $\kbt$ & \raw{\kbt}\\
    Zéro absolu & $\ok$ & \raw{\ok}\\
    <<~de Fermi~>> & $\placeholder\fermi$ & \raw{\fermi}\\
    Exponentielle & $\e{x}$ & \raw{\e{x}} & \raw{\e x}\\
    & $\eiwt$ & \raw{\eiwt}\\
    & $\emiwt$ & \raw{\emiwt}\\
    Champ électrique & $\E$ & \raw{\E}\\
    Champ électrique & $\E[x]$ & \raw{\E[x]}\\
    Champ magnétique & $\B$ & \raw{\B}\\
    Champ magnétique & $\B[x]$ & \raw{\B[x]}\\
    Gradient & $\grd$ & \raw{\grd}\\
    Divergence & $\div$ & \raw{\div}\\
    Rotationnel & $\rot$ & \raw{\rot}\\
    Rotationnel (anglais) & $\crl$ & \raw{\crl}\\
    Gradient & $\grad{}$ & \raw{\grad}\\
    Divergence & $\divg{}$ & \raw{\divg}\\
    Rotationnel & $\rota{}$ & \raw{\rota}\\
    Rotationnel (anglais) & $\curl{}$ & \raw{\curl}\\
    Espace vectoriel de matrices & $\matn{n}$ & \raw{\matn{n}} & \raw{\matn n}\\
    Espace vectoriel de matrices & $\matn[\C]{n}$ & \raw{\matn[\C]{n}} & \raw{\matn\C n}\\
    Espace vectoriel de matrices & $\matnm{n}{m}$ & \raw{\matnm{n}{m}} & \raw{\matnm nm}\\
    Espace vectoriel de matrices & $\matnm[\C]{n}{m}$ & \raw{\matnm[\C]{n}{m}} & \raw{\matnm\C nm}\\
    Identité & $\1$ & \raw{\1}\\
    Identité & $\1[n \times n]$ & \raw{\1[n \times n]}\\
    Identité & $\id$ & \raw{\id}\\
    Identité & $\id[n \times n]$ & \raw{\id[n \times n]}\\
    Identité & $\Id$ & \raw{\Id}\\
    Identité & $\Id[n \times n]$ & \raw{\Id[n \times n]}
    \end{tabular}
\end{table}

\clearpage
\subsection{Tenseurs et géométrie}

\begin{center}
    \begin{tabular}{lll}
        $\qv x$ & \raw{\qv{x}} & \raw{\qv x}\\
        $\ts\Lambda\mu\nu$ & \raw{\ts{\Lambda}{\mu}{\nu}} & \raw{\ts\Lambda\mu\nu}
    \end{tabular}
\end{center}

\begin{center}
    \begin{tabular}{lc}
        \raw{\minkp} & $\minkp$\\\\
        \raw{\minkp} & $\minkn$\\\\
        \raw{\lorentzx} & $\lorentzx$\\\\
        \raw{\lorentzy} & $\lorentzy$\\\\
        \raw{\lorentzz} & $\lorentzz$\\\\
        \raw{\lorentzr} & $\lorentzr$
    \end{tabular}
\end{center}

\subsection{Équations et fragments d'équations}

Les commandes (définies dans cette partie) commençant par \raw{\a} sont des variantes <<~alignées~>>, et doivent être utilisées dans un environnement \raw{align}, \raw{align*} ou \raw{aligned}. L'ancre d'alignement (\raw{&}) se situe juste avant le signe~\raw{=}.

\begin{adjustbox}{center}
    \centering
    {\renewcommand{\arraystretch}{2.4}\begin{tabular}{m{6cm}lp{3cm}}
        \multicolumn{1}{c}{Nom(s)} & \multicolumn{1}{c}{Résultat} & \multicolumn{1}{c}{Commande(s)}\\
        \toprule
        Équation de Schrödinger & $\eqschr$ & \raw{\eqschr}\linebreak\raw{\aeqschr}\\
        Équation de Schrödinger indépendante du temps & $\eqschrind{\psi}$ & \raw{\eqschrind{\psi}}\linebreak\raw{\aeqschrind{\psi}}\\
        Équation de Schrödinger indépendante du temps & $\eqschrind[n]{\psi}$ & \raw{\eqschrind[n]{\psi}}\linebreak\raw{\aeqschrind[n]{\psi}}\\
        Transformée de Fourier (en fréquence) & $\exprtf$ & \raw{\exprtf}\\
        Transformée de Fourier (en fréquence) & $\exprtf[g]$ & \raw{\exprtf[g]}\\
        Transformée de Fourier (en fréquence) & $\exprtu$ & \raw{\exprtu}\\
        Transformée de Fourier (en fréquence) & $\exprtu[g]$ & \raw{\exprtu[g]}\\
        Transformée de Fourier (en pulsation) & $\exprtw$ & \raw{\exprtw}\\
        Transformée de Fourier (en pulsation) & $\exprtw[g]$ & \raw{\exprtw[g]}\\
        Transformée de Fourier inverse (en\linebreak fréquence) & $\exprtfi$ & \raw{\exprtfi}\\
        Transformée de Fourier inverse (en\linebreak fréquence) & $\exprtfi[\hat{g}]$ & \raw{\exprtfi[\hat{g}]}\\
        Transformée de Fourier inverse (en\linebreak fréquence) & $\exprtui$ & \raw{\exprtui}\\
        Transformée de Fourier inverse (en\linebreak fréquence) & $\exprtui[\hat{g}]$ & \raw{\exprtui[\hat{g}]}\\
        Transformée de Fourier inverse (en\linebreak pulsation) & $\exprtwi$ & \raw{\exprtwi}\\
        Transformée de Fourier inverse (en\linebreak pulsation) & $\exprtwi[\hat{g}]$ & \raw{\exprtwi[\hat{g}]}\\
    \end{tabular}}
\end{adjustbox}

\begin{adjustbox}{center}
    \centering
    {\renewcommand{\arraystretch}{2.4}\begin{tabular}{m{6.5cm}lp{3cm}}
        \multicolumn{1}{c}{Nom(s)} & \multicolumn{1}{c}{Résultat} & \multicolumn{1}{c}{Commande(s)}\\
        \toprule
        Équation d'Euler-Lagrange & $\eqeula$ & \raw{\eqeula}\linebreak\raw{\aeqeula}\\
        Équation d'Euler-Lagrange & $\eqeula[x_i]$ & \raw{\eqeula[x_i]}\linebreak\raw{\aeqeula[x_i]}\\
        Équation de Maxwell-Ampère (locale) & $\eqma$ & \raw{\eqma}\linebreak\raw{\aeqma}\\
        Équation de Maxwell-Faraday (locale) & $\eqmf$ & \raw{\eqmf}\linebreak\raw{\aeqmf}\\
        Équation de Maxwell-Gauss (locale) & $\eqmg$ & \raw{\eqmg}\linebreak\raw{\aeqmg}\\
        \raggedright Équation de Maxwell-Thomson (locale)\linebreak Équation de Maxwell-flux (locale) & $\eqmt$ & \raw{\eqmt}\linebreak\raw{\aeqmt}\\
        Équation de Maxwell-Ampère (intégrale) & $\eqmai$ & \raw{\eqmai}\linebreak\raw{\aeqmai}\\
        Équation de Maxwell-Faraday (intégrale) & $\eqmfi$ & \raw{\eqmfi}\linebreak\raw{\aeqmfi}\\
        Équation de Maxwell-Gauss (intégrale) & $\eqmgi$ & \raw{\eqmgi}\linebreak\raw{\aeqmgi}\\
        Équation de Maxwell-Thomson (intégrale) & $\eqmti$ & \raw{\eqmti}\linebreak\raw{\aeqmti}\\
        Définition du vecteur de Poynting & $\defpoy$ & \raw{\defpoy}\linebreak\raw{\adefpoy}\\
        Théorème de Poynting & $\thmpoy$ & \raw{\thmpoy}\linebreak\raw{\athmpoy}\\
        Équation de conservation & $\eqcons$ & \raw{\eqcons}\linebreak\raw{\aeqcons}\\
        Équation de la chaleur & $\eqch$ & \raw{\eqch}\linebreak\raw{\aeqch} \\
        Équation de diffusion & $\eqdiff{c}$ & \raw{\eqdiff{c}}\linebreak\raw{\aeqdiff{c}} \\
        Équation de diffusion & $\eqdiff[c]{c}$ & \raw{\eqdiff[c]}\linebreak\raw{\aeqdiff[c]{c}}
    \end{tabular}}
\end{adjustbox}

\subsection{Calligraphie}

\begin{table}[H]
    \centering
    \begin{tabular}{ll ll ll ll ll ll ll}
    $\cala$ & \raw{\cala} &&& $\caln$ & \raw{\caln} &&& $\mcala$ & \raw{\mcala} &&& $\mcaln$ & \raw{\mcaln}\\
    $\calb$ & \raw{\calb} &&& $\calo$ & \raw{\calo} &&& $\mcalb$ & \raw{\mcalb} &&& $\mcalo$ & \raw{\mcalo}\\
    $\calc$ & \raw{\calc} &&& $\calp$ & \raw{\calp} &&& $\mcalc$ & \raw{\mcalc} &&& $\mcalp$ & \raw{\mcalp}\\
    $\cald$ & \raw{\cald} &&& $\calq$ & \raw{\calq} &&& $\mcald$ & \raw{\mcald} &&& $\mcalq$ & \raw{\mcalq}\\
    $\cale$ & \raw{\cale} &&& $\calr$ & \raw{\calr} &&& $\mcale$ & \raw{\mcale} &&& $\mcalr$ & \raw{\mcalr}\\
    $\calf$ & \raw{\calf} &&& $\cals$ & \raw{\cals} &&& $\mcalf$ & \raw{\mcalf} &&& $\mcals$ & \raw{\mcals}\\
    $\calg$ & \raw{\calg} &&& $\calt$ & \raw{\calt} &&& $\mcalg$ & \raw{\mcalg} &&& $\mcalt$ & \raw{\mcalt}\\
    $\calh$ & \raw{\calh} &&& $\calu$ & \raw{\calu} &&& $\mcalh$ & \raw{\mcalh} &&& $\mcalu$ & \raw{\mcalu}\\
    $\cali$ & \raw{\cali} &&& $\calv$ & \raw{\calv} &&& $\mcali$ & \raw{\mcali} &&& $\mcalv$ & \raw{\mcalv}\\
    $\calj$ & \raw{\calj} &&& $\calw$ & \raw{\calw} &&& $\mcalj$ & \raw{\mcalj} &&& $\mcalw$ & \raw{\mcalw}\\
    $\calk$ & \raw{\calk} &&& $\calx$ & \raw{\calx} &&& $\mcalk$ & \raw{\mcalk} &&& $\mcalx$ & \raw{\mcalx}\\
    $\call$ & \raw{\call} &&& $\caly$ & \raw{\caly} &&& $\mcall$ & \raw{\mcall} &&& $\mcaly$ & \raw{\mcaly}\\
    $\calm$ & \raw{\calm} &&& $\calz$ & \raw{\calz} &&& $\mcalm$ & \raw{\mcalm} &&& $\mcalz$ & \raw{\mcalz}
    \end{tabular}
\end{table}

\subsection{Mécanique quantique}

\subsubsection{Opérateurs et espaces de Hilbert}

\begin{table}[H]
    \centering
    \begin{tabular}{ll ll ll ll ll ll ll ll ll}
        $\hha$ & \raw{\ha} &&& $\hhn$ & \raw{\hn} &&& $\hhA$ & \raw{\hA} &&& $\hhN$ & \raw{\hN} &&& $\hhpx$ & \raw{\hhpx}\\
        $\hhb$ & \raw{\hb} &&& $\hho$ & \raw{\ho} &&& $\hhB$ & \raw{\hB} &&& $\hhO$ & \raw{\hO} &&& $\hhpy$ & \raw{\hhpy}\\
        $\hhc$ & \raw{\hc} &&& $\hhp$ & \raw{\hp} &&& $\hhC$ & \raw{\hC} &&& $\hhP$ & \raw{\hP} &&& $\hhpz$ & \raw{\hhpz}\\
        $\hhd$ & \raw{\hd} &&& $\hhq$ & \raw{\hq} &&& $\hhD$ & \raw{\hD} &&& $\hhQ$ & \raw{\hQ} &&& $\hhSx$ & \raw{\hhSx}\\
        $\hhe$ & \raw{\he} &&& $\hhr$ & \raw{\hr} &&& $\hhE$ & \raw{\hE} &&& $\hhR$ & \raw{\hR} &&& $\hhSy$ & \raw{\hhSy}\\
        $\hhf$ & \raw{\hf} &&& $\hhs$ & \raw{\hs} &&& $\hhF$ & \raw{\hF} &&& $\hhS$ & \raw{\hS} &&& $\hhSz$ & \raw{\hhSz}\\
        $\hhg$ & \raw{\hg} &&& $\hht$ & \raw{\ht} &&& $\hhG$ & \raw{\hG} &&& $\hhT$ & \raw{\hT} &&& $\hsigx$ & \raw{\hsigx}\\
        $\hhh$ & \raw{\hh} &&& $\hhu$ & \raw{\hu} &&& $\hhH$ & \raw{\hH} &&& $\hhU$ & \raw{\hU} &&& $\hsigy$ & \raw{\hsigy}\\
        $\hhi$ & \raw{\hi} &&& $\hhv$ & \raw{\hv} &&& $\hhI$ & \raw{\hI} &&& $\hhV$ & \raw{\hV} &&& $\hsigz$ & \raw{\hsigz}\\
        $\hhj$ & \raw{\hj} &&& $\hhw$ & \raw{\hw} &&& $\hhJ$ & \raw{\hJ} &&& $\hhW$ & \raw{\hW} &&& $\ham$ & \raw{\ham}\\
        $\hhk$ & \raw{\hk} &&& $\hhx$ & \raw{\hx} &&& $\hhK$ & \raw{\hK} &&& $\hhX$ & \raw{\hX} &&& $\hil$ & \raw{\hil}\\
        $\hhl$ & \raw{\hl} &&& $\hhy$ & \raw{\hy} &&& $\hhL$ & \raw{\hL} &&& $\hhY$ & \raw{\hY} &&& \\
        $\hhm$ & \raw{\hm} &&& $\hhz$ & \raw{\hz} &&& $\hhM$ & \raw{\hM} &&& $\hhZ$ & \raw{\hZ} &&&
    \end{tabular}
\end{table}

\subsubsection{Matrices de Pauli}

\begin{align*}
    \raw{\sigx} &= \sigx\\
    \raw{\sigy} &= \sigy\\
    \raw{\sigz} &= \sigz\\
    \raw{\sigmaw} &= \sigmaw\\
    \raw{\sigmax} &= \sigmax\\
    \raw{\sigmay} &= \sigmay\\
    \raw{\sigmaz} &= \sigmaz
\end{align*}

\clearpage
\section{Code source}

\subsection{Code source en ligne}

\begin{center}
    \begin{tabular}{lll}
        Langage(s) & Résultat & Commande\\
        \toprule
        \textbf{Tous} & \code{python}{print("hello world")} & \raw{\code{python}{print("hello world")}}\\
        \midrule
        C, C$++$ & \cpp{printf("hello world");} & \raw{\cpp{printf("hello world");}}\\
        Java & \java{System.out.println("hello world");} & \raw{\java{System.out.println("hello world");}}\\
        Javascript & \js{console.log("hello world");} & \raw{\js{console.log("hello world");}}\\
        Python & \py{print("hello world")} & \raw{\py{print("hello world")}}\\
    \end{tabular}
\end{center}

\subsection{Code source en bloc}

\subsubsection{Écrire du code directement dans le document \LaTeX}

\begin{minipage}[t]{0.475\linewidth}
    \begin{minted}{python}
def factorielle(n):
    if n == 0:
        return 1
    return n * factorielle(n - 1)
    \end{minted}
\end{minipage}\hfill\begin{minipage}[t]{0.475\textwidth}
    \inputminted{latex}{codeblock.tex}
\end{minipage}

\subsubsection{Inclure le code source d'un autre document}

\begin{minipage}[t]{0.475\linewidth}
    \inputminted{latex}{codeblock.tex}
\end{minipage}\hfill\begin{minipage}[t]{0.475\textwidth}
    \code{latex}{\inputminted{latex}{codeblock.tex}}
\end{minipage}

\clearpage
\section{Algorithmes}

\begin{algorithm}[H]
\Entree{tableau 2D $M$ de taille $n \times n$ à valeurs dans $\qty{0, 1}$}
\Sortie{indices $(i_0, j_0) \in \llbracket 1, n \rrbracket^2$ du coin inférieur droit et longueur $n_0$ des côtés d'un carré de $0$ de taille maximale dans l'image $M$}
\Deb{
$F \gets \text{tableau 2D de taille $n \times n$ rempli de $0$}$\\
$(i_0, j_0, n_0) \gets (0, 0, 0)$\\
\PourCh{$i \in \llbracket 1, n \rrbracket$}{
    \PourCh{$j \in \llbracket 1, n \rrbracket$}{
        \Si{$M[i, j] = 0$}{
            $F[i, j] \gets 1 + \min(F[i-1, j], F[i, j-1], F[i-1, j-1])$\\
            \Si{$F[i, j] > n_0$}{
                $n_0 \gets F[i, j]$\\
                $(i_0, j_0) \gets (i, j)$
            }
        }
    }
}
\Retour{$(i_0, j_0)$, $n_0$}
}
\caption{Algorithme CarréMaximal.}
\end{algorithm}

\begin{minted}{latex}
\begin{algorithm}[H]
\Entree{tableau 2D $M$ de taille $n \times n$ à valeurs dans $\qty{0, 1}$}
\Sortie{indices $(i_0, j_0) \in \llbracket 1, n \rrbracket^2$ du coin inférieur droit et longueur $n_0$ des côtés d'un carré de $0$ de taille maximale dans l'image $M$}
\Deb{
$F \gets \text{tableau 2D de taille $n \times n$ rempli de $0$}$\\
$(i_0, j_0, n_0) \gets (0, 0, 0)$\\
\PourCh{$i \in \llbracket 1, n \rrbracket$}{
    \PourCh{$j \in \llbracket 1, n \rrbracket$}{
        \Si{$M[i, j] = 0$}{
            $F[i, j] \gets 1 + \min(F[i-1, j], F[i, j-1], F[i-1, j-1])$\\
            \Si{$F[i, j] > n_0$}{
                $n_0 \gets F[i, j]$\\
                $(i_0, j_0) \gets (i, j)$
            }
        }
    }
}
\Retour{$(i_0, j_0)$, $n_0$}
}
\caption{Algorithme CarréMaximal.}
\end{algorithm}
\end{minted}

\clearpage
\section{Code source de ce document}
\inputminted[fontsize=\small]{latex}{enssdm.tex}

\clearpage
\section{Code source du package \raw{enssdm}}
\inputminted[fontsize=\small]{latex}{enssdm.sty}

\end{document}